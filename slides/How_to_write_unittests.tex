\documentclass[SoftwareQuality.tex]{subfiles}
\begin{document}

\section{required tooling}
\begin{frame}{starter kit for Java developers} 
	\begin{itemize}
		\item IDE including testplugin (eclipse + infinitest / Netbeans + ?)  
		\item build -- tool (maven / gradle {\color{gray}/ ant})
		\item SCM (git  {\color{gray}/ subversion})
		\item testing framework (JUnit / NUnit)
		\item mocking framwork (Mockito)
	\end{itemize}
\end{frame}


\defverbatim[colored]\makeset{
\begin{lstlisting}[ basicstyle=\tiny]
class PlainOldJavaClass{
   @Test // marks a method so that its been called by JUnit
   public // test methods must be public
   void  // test methods must not return anything
   testedMethod__preconditions__expectedBehavior() // no parameters
   {
     // arrange: create and configure dependencies and variables
     int input = 4;
     int expectedResult = 2;
     
     // act: create tested unit (if possible)  and call tested method.
     int calculatedResult = (int) Math.sqr(input);
     
     // assert: check the Result by calling one of JUnits assert methods.
     // the assert method throws an exception when the check fails.
     assertEquals("square root of 4", // always describe what you check.
                  expectedResult,
                  calculatedResult);  
   }
}
\end{lstlisting}
}
\begin{frame}{Writing a JUnit Test} 
\makeset
\end{frame}
\defverbatim[colored]\makeset{
\begin{lstlisting}[ basicstyle=\tiny]
class PlainOldJavaClass{
   @Test 
   public void  testedMethod__preconditions__expectedBehavior() {
     // create mock
     MyInterfaceOrClass mockObject = mock(MyInterfaceOrClass.class);
     //create a spy     
     MyClass spyObject = spy(new MyClass());
     // configure behavior:
     doThrow(new RuntimeException("should not be called")).when(mockObject).anyMethod();
     doReturn(mockObject).thenReturn(null).when(spyObject).getMyInterfaceOrClass();
     
     // alternative for non void methods:
     .when ( spyObject.getMyInterfaceOrClass() ) .thenReturn(mockObject,null,mockObject);     
     
     // check call of method
     verify(spyObject).expectedMethodCall(mockObject,any(OtherInterfaceOrClass.class));
   }
}
\end{lstlisting}
}
\begin{frame}{Using Mockito} 
\makeset
\end{frame}

\begin{frame}{Test Driven Development} 
When developing your software follow this cycle:
\vfill
\begin{itemize}
\item {\color{red}Test - write enough of a single test so that it fails.\\
   {\footnotesize not compiling means fail...}}
\vfill
\item {\color{green}Implement - change the production code to make the test compile and pass.\\
   {\footnotesize Use the simplest possible solution.}}
\vfill
\item {\color{green}Refactor - change the production code to make it ''clean''.\\
   {\footnotesize Resolve code duplications, introduce design patterns, but no test must break at the end.\\Also refactor your test code.}}
\vfill
\end{itemize}


\end{frame}
\end{document}